% Convenience functions
\newcommand{\pfrac}[2]
        {\frac{\partial #1}{\partial #2}}       % partial 1 /partial 2 
\newcommand{\ket}[1]
        {\left | \, #1 \right \rangle}          % bra-ket notation
\newcommand{\abs}[1]{
        \left\vert #1 \right\vert}		% vert bars for averages
\newcommand{\norm}[1]
        {\left\Vert #1 \right\Vert}             % taller vert bars for the norm
\newcommand{\evalat}[1]
        {\left. #1 \right \vert}        % ex. evaluating the int. at its limits
\newcommand{\set}[1]
        {\left\{ #1 \right\}}           % squigle brackets for sets
\newcommand{\avg}[1]
        {\left< #1 \right>}		% angle brackets for averages < >
\newcommand{\paren}[1]
        {\left( #1 \right)}		% grows parentheses () 
\newcommand{\brackets}[1]
        {\left[ #1 \right]}		% grows square brackets []
\newcommand{\braces}[1]
        {\left \{ #1 \right \}}         % grows curly brackets {}
\newcommand{\piecewisebrace}[1]
        {\left \{ #1 \right .}          % piecewisebrace

% Shorthand with proper spacing
\newcommand{\viz}{viz.\ }
\newcommand{\ie}{i.e.\ }
\newcommand{\eg}{e.g.\ }
\newcommand{\cf}{c.f.\ }

% Chemical formula
\newcommand{\chem}[1]{\ensuremath{\mathrm{#1}}} 

% New math operators
\DeclareMathOperator*{\minor}{minor\,}
\DeclareMathOperator*{\sgn}{\mathbf{sgn}\, }
\DeclareMathOperator*{\rootof}{RootOf\,}

% Vector notation
\newcommand{\mathvect}[1]{\boldsymbol{#1}}
\newcommand{\vect}[1]{\mathbf{#1}}

% Differential used at the end of an integral
\newcommand*\diff{\mathop{}\!\mathrm{d}}
\newcommand{\integral}[2]{\int #1\diff#2}
