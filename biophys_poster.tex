\documentclass[landscape]{sciposter}
%\documentclass[landscape,draft]{sciposter}
\usepackage[]{graphicx}
\usepackage{amsmath,amsfonts,amssymb}
\usepackage{units}
\usepackage{authoraftertitle}
\usepackage[utf8]{inputenc}
\usepackage[T1]{fontenc}
\usepackage{charter}
\usepackage[expert]{mathdesign}
%\usepackage{fancyhdr}
%\usepackage{etoolbox}
%\renewcommand{\familydefault}{\rmdefault}

\usepackage{multicol}
\usepackage{url}

% Set the lengths of the pictures
\newlength{\customfigwidth}
\setlength{\customfigwidth}{16cm}

\setlength{\belowdisplayskip}{10pt} \setlength{\belowdisplayshortskip}{10pt}
\setlength{\abovedisplayskip}{10pt} \setlength{\abovedisplayshortskip}{10pt}
\setlength{\columnseprule}{4pt}
\setlength{\multicolsep}{6.0pt plus 2.0pt minus 1.5pt}% 50% of original values
\setlength\columnsep{40pt}% 50% of original values

\newcommand{\eqnote}[1]{{\scriptsize#1}}


\usepackage[font={large,it}]{caption}
\usepackage[protrusion=true,expansion=true]{microtype}

% Convenience functions
\newcommand{\pfrac}[2]
        {\frac{\partial #1}{\partial #2}}       % partial 1 /partial 2 
\newcommand{\ket}[1]
        {\left | \, #1 \right \rangle}          % bra-ket notation
\newcommand{\abs}[1]{
        \left\vert #1 \right\vert}		% vert bars for averages
\newcommand{\norm}[1]
        {\left\Vert #1 \right\Vert}             % taller vert bars for the norm
\newcommand{\evalat}[1]
        {\left. #1 \right \vert}        % ex. evaluating the int. at its limits
\newcommand{\set}[1]
        {\left\{ #1 \right\}}           % squigle brackets for sets
\newcommand{\avg}[1]
        {\left< #1 \right>}		% angle brackets for averages < >
\newcommand{\paren}[1]
        {\left( #1 \right)}		% grows parentheses () 
\newcommand{\brackets}[1]
        {\left[ #1 \right]}		% grows square brackets []
\newcommand{\braces}[1]
        {\left \{ #1 \right \}}         % grows curly brackets {}
\newcommand{\piecewisebrace}[1]
        {\left \{ #1 \right .}          % piecewisebrace

% Shorthand with proper spacing
\newcommand{\viz}{viz.\ }
\newcommand{\ie}{i.e.\ }
\newcommand{\eg}{e.g.\ }
\newcommand{\cf}{c.f.\ }

% Chemical formula
\newcommand{\chem}[1]{\ensuremath{\mathrm{#1}}} 

% New math operators
\DeclareMathOperator*{\minor}{minor\,}
\DeclareMathOperator*{\sgn}{\mathbf{sgn}\, }
\DeclareMathOperator*{\rootof}{RootOf\,}

% Vector notation
\newcommand{\mathvect}[1]{\boldsymbol{#1}}
\newcommand{\vect}[1]{\mathbf{#1}}

% Differential used at the end of an integral
\newcommand*\diff{\mathop{}\!\mathrm{d}}
\newcommand{\integral}[2]{\int #1\diff#2}


% Equation placeholders
\newcommand{\SPHYvals}{%
  Y_{\ell m}^{*} ( \hat{\vect{r}}_i ) 
  Y_{\ell m}     ( \hat{\vect{r}} ) 
}

\newcommand{\SPHsumterms}{%
    \sum_{\ell=0}^\infty \sum_{m=-\ell}^{\ell} %
}

\newcommand{\SPHsum}[1]{%
  \SPHsumterms
  #1
  \SPHYvals
}

\newcommand{\normdiff}[1]{%
  \norm{ \vect{#1} - \vect{#1}_i}
}

\newcommand{\ViralInteractionPot}{\Phi_{12}}
\newcommand{\transpose}[1]{^{\intercal}}

\newcommand{\sphericalBesselIN}{ i }
\newcommand{\sphericalBesselKN}{ k }
\newcommand{\knownPHI}{\Psi}
\newcommand{\macroPHI}{\Phi}

\newcommand{\pka}{\text{pKa}}
\newcommand{\pH}{\text{pH}}
\newcommand{\LMAX}{\ell_{\text{MAX}}}

\newcommand{\effChargeLocations}{\vect{r}_1, \vect{r}_2, \ldots, \vect{r}_N}
\newcommand{\fittingACCeffective}{\chi}
\newcommand{\debyehukel}{Debye-H\"{u}ckel }

%\definecolor{BoxCol}{rgb}{.93531, .93531, .93531}
%\definecolor{palered}{RGB}{255,182,182}


\newcommand{\titlefont}[1]{%
  {%
  \usefont{T1}{bch}{b}{n}%
    \selectfont%
    #1%
  }%
  \normalfont%
}

\newcommand{\at}{\makeatletter\titlefont{@}\makeatother}


\author{Travis Hoppe, Robert Best}
\email{hoppeta@mail.nih.gov}
\institute{National Institutes of Health, National Institute of Diabetes and Digestive and Kidney Diseases}
\conference{\titlefont{\large Biophysical Society Meeting 2016, Los Angeles CA}}

\renewcommand{\sectionsize}{%
  \usefont{T1}{bch}{b}{n}%
  \scshape%
  \bfseries%
  \large
  \selectfont%
}
 
\begin{document}

% Custom footer
\renewcommand{\footlogo}{%
  \resizebox{2\logowidth}{!}{%
    \includegraphics[height = 1cm]{logos/NIDDK_logo.pdf}%
    \hspace{1em}
    \includegraphics[height = 1cm]{logos/dhhs_logo.pdf}%
    \hspace{1em}
    \includegraphics[height = 1cm]{logos/NIH_Logo.pdf}
  }
}

% Custom title
%\maketitle
\resizebox{\textwidth}{!}{
  \begin{tabular}{ p{.6\textwidth} r }
    \titlefont{\Huge  ENHANCING THE COEVOLUTIONARY SIGNAL }  &
    \titlefont{\Large \MyAuthor} \\
    %& 
    \titlefont{\Huge }&
    %\includegraphics[height = 1cm]{mail.png}
    \texttt{hoppeta}{\bfseries \at}\texttt{mail.nih.gov}
    %&
  \end{tabular}
}
\vspace{1cm}


\begin{multicols}{3}

\begin{abstract}
Effective coarse-grained representations of protein-protein interaction potentials are vital in the modeling of large scale systems.
We develop a method to fit an arbitrary number of effective charges to approximate the electrostatic potential of a protein at a given pH in an ionic solution.
We find that the effective charges can reproduce an input potential calculated from a high resolution Poisson-Boltzmann calculation. 
The fitting procedure uses a number of approximations in the charge magnitudes, initial conditions and multipoles to speed convergence.
The most significant gains are found by fitting the multipole moments of the effective charge potential to the moments of the original field.
We compute interaction energies and find excellent agreement to the original potential.
From the effective charge model we compute the electrostatic contribution to the second virial coefficient.
\end{abstract}

\columnbreak

\section*{Methods}
%
Calculate the electrostatic potential from the non-linear Poisson-Boltzmann equation
\begin{equation}
  \label{eq:poisson_boltzmann}
  -\nabla \cdot \epsilon( \vect{r} ) \nabla \Psi (\vect{r}) + 
  \kappa^2 \sinh \Psi (\vect{r}) 
  = \rho(\vect{r})
\end{equation}
\eqnote{with $\Psi$ as the potential, $\rho$ as the charge density and $\kappa$ as the inverse Debye length. Calculated with APBS.}
\vspace{.1em}

Approximate the potential as an expansion of a screened electrostatic potential and truncate the series to a finite order $\LMAX$
\begin{equation}
  \label{eq:yukawa_coeff}
  \Psi( \vect{r} ) = 
  \sum_{\ell=0}^{\LMAX} \sum_{m=-\ell}^{\ell}
  \alpha_{\ell m} \sphericalBesselKN(\kappa r) Y_{\ell m} (\theta, \phi)
\end{equation}
\eqnote{Where $Y_{\ell,m}$, $\sphericalBesselIN$, $\sphericalBesselKN$ are the spherical harmonics and the modified spherical Bessel functions.}
\vspace{.1em}

The multipole coefficients $\alpha$ are defined as a sum over the locations of the charges
\section*{Methods cont.}
\vspace{-1em}

\begin{equation}
  \label{eq:yukawa_coeff_inv}
  \alpha_{\ell m} = 8 \kappa \sum_{\vect{r}_i=(r'_i,\theta'_i,\phi'_i)}^N
  q_i \sphericalBesselIN_\ell(\kappa r'_i) Y_{\ell, -m}(\theta'_i, \phi'_i)
\end{equation}
\eqnote{valid for all points outside a spherical shell enclosing the charges}
\vspace{.1em}

Find a set of $N$ effective charges that reproduce $\Psi$ from a pairwise Yukawa potential $\Phi$
\begin{equation}
  \label{eq:yukawa}
  \Phi(\vect{r}) = \sum_{i, j \neq i}^N
  q_i 
  \frac
  {e^{-\kappa \normdiff{r}}}
  {\normdiff{r}}
\end{equation}

Instead of fitting $\Psi \rightarrow \Phi$, calculate the multipole coefficients of the effective charges and fit those to the multipole coefficients of the original field. For a complete description along with references please see attached paper.


%\columnbreak

%\renewcommand{\refname}{} 
%\bibliographystyle{plain}  % enter bibligraphy as usual, with or without using
%\small                     
%\bibliography{../bib/AnisoChargeFits.bib} 


%\renewcommand{\refname}{xxx} 
%\begin{sectionbox}{}  
% changes section heading over bibliography
%\bibliographystyle{unsrt}
%\bibliography{../bib/AnisoChargeFits.bib}
%\end{sectionbox}

\end{multicols}


\begin{multicols}{3}

  \section*{Human Serum Albumin}

%
\vfill
\columnbreak


\section*{Ovalbumin}
%

%
\vfill
\columnbreak
\section*{Lysozme}
%


%
\vfill
\end{multicols}

\begin{multicols}{3}

  a

  \columnbreak

  b

  \columnbreak

  c

\end{multicols}



\begin{multicols}{3}


\section*{Effective Charge Fit Quality}
%

%
\vfill
\columnbreak

\section*{Interaction Energy}
%
%\begin{figure}[p]
%  \center
%  \includegraphics[scale=1.2]
%  {eff_vs_APBS_all-crop}
% \caption[]{
%   Interaction energy of two identical molecules separated by a distance of $1.%25$ protein diameters along the x-axis. 
%Even at the isoelectric point (where monopole moment vanishes), the effective c%harges do a great job of reproducing the interaction potential.
%The interaction energy, calculated with APBS, is shown as dashed black line. 
%Each APBS configuration took several hours.
%In contrast, the computational cost of the effective charge model was negligible.
%}
% \label{fig:interaction_energy_oval}
%\begin{align}
%  \notag
%  \Delta \Delta U_{\mathvect{\phi}} =
%  \Delta U_{\mathvect{\phi}}^{\text{complex}} -
%  \Delta U_{\mathvect{\phi}}^{\text{p1}} -
%  \Delta U_{\mathvect{\phi}}^{\text{p2}}
%\end{align}
%\end{figure}
%
\vfill
\columnbreak

\section*{Second Virial Coefficient $B_{22}$}
%

%
\vfill
\end{multicols}




 
\end{document}
