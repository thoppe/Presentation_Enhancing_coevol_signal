\documentclass[landscape]{sciposter}
%\documentclass[landscape,draft]{sciposter}
\usepackage[]{graphicx}
\usepackage{amsmath,amsfonts,amssymb}
\usepackage{units}
\usepackage{authoraftertitle}
\usepackage[utf8]{inputenc}
\usepackage[T1]{fontenc}
\usepackage{charter}
\usepackage[expert]{mathdesign}
%\usepackage{fancyhdr}
%\usepackage{etoolbox}
%\renewcommand{\familydefault}{\rmdefault}

\usepackage{multicol}
\usepackage{url}

% Set the lengths of the pictures
\newlength{\customfigwidth}
\setlength{\customfigwidth}{16cm}

\setlength{\belowdisplayskip}{10pt} \setlength{\belowdisplayshortskip}{10pt}
\setlength{\abovedisplayskip}{10pt} \setlength{\abovedisplayshortskip}{10pt}
\setlength{\columnseprule}{4pt}
\setlength{\multicolsep}{6.0pt plus 2.0pt minus 1.5pt}% 50% of original values
\setlength\columnsep{40pt}% 50% of original values

\newcommand{\eqnote}[1]{{\scriptsize#1}}


\usepackage[font={large,it}]{caption}
\usepackage[protrusion=true,expansion=true]{microtype}

% Convenience functions
\newcommand{\pfrac}[2]
        {\frac{\partial #1}{\partial #2}}       % partial 1 /partial 2 
\newcommand{\ket}[1]
        {\left | \, #1 \right \rangle}          % bra-ket notation
\newcommand{\abs}[1]{
        \left\vert #1 \right\vert}		% vert bars for averages
\newcommand{\norm}[1]
        {\left\Vert #1 \right\Vert}             % taller vert bars for the norm
\newcommand{\evalat}[1]
        {\left. #1 \right \vert}        % ex. evaluating the int. at its limits
\newcommand{\set}[1]
        {\left\{ #1 \right\}}           % squigle brackets for sets
\newcommand{\avg}[1]
        {\left< #1 \right>}		% angle brackets for averages < >
\newcommand{\paren}[1]
        {\left( #1 \right)}		% grows parentheses () 
\newcommand{\brackets}[1]
        {\left[ #1 \right]}		% grows square brackets []
\newcommand{\braces}[1]
        {\left \{ #1 \right \}}         % grows curly brackets {}
\newcommand{\piecewisebrace}[1]
        {\left \{ #1 \right .}          % piecewisebrace

% Shorthand with proper spacing
\newcommand{\viz}{viz.\ }
\newcommand{\ie}{i.e.\ }
\newcommand{\eg}{e.g.\ }
\newcommand{\cf}{c.f.\ }

% Chemical formula
\newcommand{\chem}[1]{\ensuremath{\mathrm{#1}}} 

% New math operators
\DeclareMathOperator*{\minor}{minor\,}
\DeclareMathOperator*{\sgn}{\mathbf{sgn}\, }
\DeclareMathOperator*{\rootof}{RootOf\,}

% Vector notation
\newcommand{\mathvect}[1]{\boldsymbol{#1}}
\newcommand{\vect}[1]{\mathbf{#1}}

% Differential used at the end of an integral
\newcommand*\diff{\mathop{}\!\mathrm{d}}
\newcommand{\integral}[2]{\int #1\diff#2}


% Equation placeholders
\newcommand{\SPHYvals}{%
  Y_{\ell m}^{*} ( \hat{\vect{r}}_i ) 
  Y_{\ell m}     ( \hat{\vect{r}} ) 
}

\newcommand{\SPHsumterms}{%
    \sum_{\ell=0}^\infty \sum_{m=-\ell}^{\ell} %
}

\newcommand{\SPHsum}[1]{%
  \SPHsumterms
  #1
  \SPHYvals
}

\newcommand{\normdiff}[1]{%
  \norm{ \vect{#1} - \vect{#1}_i}
}

\newcommand{\ViralInteractionPot}{\Phi_{12}}
\newcommand{\transpose}[1]{^{\intercal}}

\newcommand{\sphericalBesselIN}{ i }
\newcommand{\sphericalBesselKN}{ k }
\newcommand{\knownPHI}{\Psi}
\newcommand{\macroPHI}{\Phi}

\newcommand{\pka}{\text{pKa}}
\newcommand{\pH}{\text{pH}}
\newcommand{\LMAX}{\ell_{\text{MAX}}}

\newcommand{\effChargeLocations}{\vect{r}_1, \vect{r}_2, \ldots, \vect{r}_N}
\newcommand{\fittingACCeffective}{\chi}
\newcommand{\debyehukel}{Debye-H\"{u}ckel }

%\definecolor{BoxCol}{rgb}{.93531, .93531, .93531}
%\definecolor{palered}{RGB}{255,182,182}


\newcommand{\titlefont}[1]{%
  {%
  \usefont{T1}{bch}{b}{n}%
    \selectfont%
    #1%
  }%
  \normalfont%
}

\newcommand{\at}{\makeatletter\titlefont{@}\makeatother}


\author{Travis Hoppe, Robert Best}
\email{hoppeta@mail.nih.gov}
\institute{National Institutes of Health, National Institute of Diabetes and Digestive and Kidney Diseases}
\conference{\titlefont{\large Biophysical Society Meeting 2016, Los Angeles CA}}

\renewcommand{\sectionsize}{%
  \usefont{T1}{bch}{b}{n}%
  \scshape%
  \bfseries%
  \large
  \selectfont%
}
 
\begin{document}

% Custom footer
\renewcommand{\footlogo}{%
  \resizebox{2\logowidth}{!}{%
    \includegraphics[height = 1cm]{logos/NIDDK_logo.pdf}%
    \hspace{1em}
    \includegraphics[height = 1cm]{logos/dhhs_logo.pdf}%
    \hspace{1em}
    \includegraphics[height = 1cm]{logos/NIH_Logo.pdf}
  }
}

% Custom title
%\maketitle
\resizebox{\textwidth}{!}{
  \begin{tabular}{ p{.6\textwidth} r }
    \titlefont{\Huge  Enhancing the Coevolutionary Signal }  &
    \titlefont{\Large \MyAuthor} \\
    %& 
    \titlefont{\Huge }&
    %\includegraphics[height = 1cm]{mail.png}
    \texttt{hoppeta}{\bfseries \at}\texttt{mail.nih.gov}
    %&
  \end{tabular}
}
\vspace{1cm}


\begin{multicols}{3}

\section*{Abstract}
x
\columnbreak

\section*{Methods}
x
\columnbreak

\section*{GREMLIN}
x
\end{multicols}

\begin{multicols}{3}
\section*{Hypothesis}
x
\columnbreak

\section*{ROC Curve}
x
\columnbreak

\section*{Improved Contact Maps}
x
\columnbreak
\end{multicols}

\begin{multicols}{3}

\section*{Q-values plots}
x
\columnbreak

\section*{Folding improvement}
x
\columnbreak

\section*{Gauss Kernels}
x
\columnbreak
\end{multicols}



%
%\begin{figure}[p]
%  \center
%  \includegraphics[scale=1.2]
%  {eff_vs_APBS_all-crop}
% \caption[]{
%   Interaction energy of two identical molecules separated by a distance of $1.%25$ protein diameters along the x-axis. 
%Even at the isoelectric point (where monopole moment vanishes), the effective c%harges do a great job of reproducing the interaction potential.
%The interaction energy, calculated with APBS, is shown as dashed black line. 
%Each APBS configuration took several hours.
%In contrast, the computational cost of the effective charge model was negligible.
%}
% \label{fig:interaction_energy_oval}
%\begin{align}
%  \notag
%  \Delta \Delta U_{\mathvect{\phi}} =
%  \Delta U_{\mathvect{\phi}}^{\text{complex}} -
%  \Delta U_{\mathvect{\phi}}^{\text{p1}} -
%  \Delta U_{\mathvect{\phi}}^{\text{p2}}
%\end{align}
%\end{figure}
%



%\renewcommand{\refname}{} 
%\bibliographystyle{plain}  % enter bibligraphy as usual, with or without using
%\small                     
%\bibliography{../bib/AnisoChargeFits.bib} 


%\renewcommand{\refname}{xxx} 
%\begin{sectionbox}{}  
% changes section heading over bibliography
%\bibliographystyle{unsrt}
%\bibliography{../bib/AnisoChargeFits.bib}
%\end{sectionbox}
 
\end{document}
